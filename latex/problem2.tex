\section{Public Key Encryption}

Note: all codes in this section include a common header file, \texttt{Part1/p2\_common.h}, which can be found in Appendix \ref{code:2_common}.

%%%%%%%%%%%%%%%%%%%%%%%%%%%%%%%%%%%%%%%%%%%%%%%%%%%%%%%%%%%%%%%%%%%%%%%%%%%%%%%%
% \subsection{Background}
% \subsection{BIGNUM APIs}
\setcounter{subsection}{2}
%%%%%%%%%%%%%%%%%%%%%%%%%%%%%%%%%%%%%%%%%%%%%%%%%%%%%%%%%%%%%%%%%%%%%%%%%%%%%%%%
\subsection{Deriving the Private Key}

The algorithm to derive private exponent $d$ is stated as Algorithm \ref{alg:rsa_pqe_d}.
Command and output is screenshot as Fig.\ref{fig:p2_3}.
Code can be found in Appendix \ref{code:2_3}

\begin{algorithm}
\caption{Calculate private key exponent $d$}
\label{alg:rsa_pqe_d}
\begin{algorithmic}
\STATE \textbf{Input:} Private primes $p$,$q$ and public exponent $e$
\STATE \textbf{Output:} Private exponent $d$

\STATE $ n \gets pq $
\STATE $ \phi(n) \gets (p-1)(q-1) $
\STATE $ d \gets e^{-1} \mod{\phi(n)} $
\RETURN $ d $
\end{algorithmic}
\end{algorithm}

\begin{figure}[t!]
\centering
\includegraphics[width=\columnwidth]{pictures/p2_3.png}
\caption{
    Calculate private key exponent $d$
}
\label{fig:p2_3}
\end{figure}

%%%%%%%%%%%%%%%%%%%%%%%%%%%%%%%%%%%%%%%%%%%%%%%%%%%%%%%%%%%%%%%%%%%%%%%%%%%%%%%%
\subsection{Encrypting a Message}

The algorithm to encrypt a message $m$ using RSA algorithm with public key $(n, e) $ is stated as Algorithm \ref{alg:rsa_enc}.
Command and output is screenshot as Fig.\ref{fig:p2_4}.
Code can be found in Appendix \ref{code:2_4}

\begin{algorithm}
\caption{RSA encrypt}
\label{alg:rsa_enc}
\begin{algorithmic}
\STATE \textbf{Input:} RSA public key $(n, e)$ and message $m$
\STATE \textbf{Output:} Ciphertext $c$

\STATE $ c \gets m^e \mod{n} $
\RETURN $ c $
\end{algorithmic}
\end{algorithm}

\begin{figure}[ht]
\centering
\includegraphics[width=\columnwidth]{pictures/p2_4.png}
\caption{
    RSA encrypt
}
\label{fig:p2_4}
\end{figure}

%%%%%%%%%%%%%%%%%%%%%%%%%%%%%%%%%%%%%%%%%%%%%%%%%%%%%%%%%%%%%%%%%%%%%%%%%%%%%%%%
\subsection{Decrypting a Message}

The algorithm to decrypt a ciphertext $c$ using RSA algorithm with private key $(n, d) $ is stated as Algorithm \ref{alg:rsa_dec}.
Command and output is screenshot as Fig.\ref{fig:p2_5}.
Code can be found in Appendix \ref{code:2_5}

\begin{algorithm}
\caption{RSA encrypt}
\label{alg:rsa_dec}
\begin{algorithmic}
\STATE \textbf{Input:} RSA private key $(n, d)$ and ciphertext $c$
\STATE \textbf{Output:} Plaintext $m$

\STATE $ m \gets c^d \mod{n} $
\RETURN $ m $
\end{algorithmic}
\end{algorithm}

\begin{figure}[ht]
\centering
\includegraphics[width=\columnwidth]{pictures/p2_5.png}
\caption{
    RSA decrypt
}
\label{fig:p2_5}
\end{figure}

%%%%%%%%%%%%%%%%%%%%%%%%%%%%%%%%%%%%%%%%%%%%%%%%%%%%%%%%%%%%%%%%%%%%%%%%%%%%%%%%
\subsection{Signing a Message}
\label{subs:sign}

The algorithm to sign a message $m$ using RSA algorithm is completely the same as RSA decryption (Algorithm \ref{alg:rsa_dec}) which utilize the private key.
Command and output is screenshot as Fig.\ref{fig:p2_6}.
Code can be found in Appendix \ref{code:2_6}

Note that when I changed \texttt{\$2000} to \texttt{\$3000}, the signature becomes different. Concretely, we changed 1 bit in the message. Then there are 123 different bits (in the total 256 bits) in the two signatures, which is nearly 50\% of all bits.

\begin{figure}[ht]
\centering
\includegraphics[width=\columnwidth]{pictures/p2_6.png}
\caption{
    RSA sign
}
\label{fig:p2_6}
\end{figure}

%%%%%%%%%%%%%%%%%%%%%%%%%%%%%%%%%%%%%%%%%%%%%%%%%%%%%%%%%%%%%%%%%%%%%%%%%%%%%%%%
\subsection{Verifying a Signature}

The algorithm to verify a signature using RSA algorithm is completely the same as RSA encryption (Algorithm \ref{alg:rsa_enc}) which utilize the public key.
Command and output is screenshot as Fig.\ref{fig:p2_7}.
Code can be found in Appendix \ref{code:2_7}

From this example, it can be seen that 1 bit different in the two signature caused the recovered measurements completely different. They even don't have the same length.

This example and the one in Question \ref{subs:sign} show that there's a great diffusion property in RSA.

\begin{figure}[ht]
\centering
\includegraphics[width=\columnwidth]{pictures/p2_7.png}
\caption{
    RSA verify
}
\label{fig:p2_7}
\end{figure}